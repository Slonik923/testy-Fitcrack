\chapter{Záver}
Cieľom práce, ako je to spomenuté už v úvode nie je plne otestovať systém Fitcrack, keďže je to rozsiahli systém skladajúci sa z niekoľkých modulov, ktoré sú stále vo vývoji, ale položiť otestovať základnú funkcionalitu a pripraviť vývojár prostredie a návod na písanie ďalších testovacích prípadov.
Veľká čast práce je preto zameraná na teóriu testovania, konkrétne na testovanie založenom na požiadavkách.
Na implementáciu testov bol použitý jazyk \texttt{Python3} a jeho framework \texttt{unittest}.
Výsledky pokrytia zdrojových kódov nie je možné určiť, keďže moduly sú súčasťou platformy Boinc.

Asimilátor je asi najjednoduchší modul a bolo možné dosiahnú 100-percentné pokrytie všetkých primárnych ciest (PPC), no cesty ako aj graf vychádzajú iba zo pseudokódu, ktorý zjednodušuje niektoré časti reálnej implementácie.
Je preto na zváženie, či by mali byť na testovanie využíté zdrojové kódy, čo by neúnosne zvýšilo náročnosť testovania, alebo o ktoré časti by mal byť pseudokód doplnenný a následne analogicky prerobené aj primárne cesty a aj samotná testovacia sada.

Generátor je podstatne zložitejší modul ako asimilátor a tomu aj odpovedá jeho pseudokód.
Implementovať testovaciu sadu, ktorá by mala pokryť 777 primárnych ciest nebolo v mojích silách, preto som zvolil iba pokrytie všetkých uzlov (node coverage).
Toto krytérium bolo splnené na 100\%, no opäť sú v pseudokóde vynechané niektoré podstatné časti.
Generátor je natoľko zložitý, že iba jeho testovanie by mohla byť práca na niekoľko mesiacov, keďže napríkad adaptatívne plánuje veľkosť úlohy pre každého hosťa, preto by sa na jeho testovaní mal primárne podieľať aj jeho autor, včom sa mu táto práca snaží pomôcť.

Runner je multiplatfrmová aplikácia, ktorá ovláda nástroj na lámanie hashov -- Hashcat.
Pre jeho testovanie bolo potrebné implemetovať dvojníka Hashcat-u, ktorý sa bude správať tak, ako to testy potrebujú.
K modulu runner neexistuje žiadna špecifikácia, takže bolo problematické vymyslieť požiadavky pre tety, no nakoniec bolo zvolené rozdielenie vstupov na triedy. 
Toto riešenie pokrýva všetky momentálne implementované možnosti.

Bolo testované aj aplikačné rozhranie, ktoré bolo vyvíjané paralelne s touto prácou a nahrádza staré webové rozhranie webAdmin. 
API nebolo otestované celé, ale iba časti, ktoré zabezpečujú základnú funkcionalitu systému Fitcrack, väčšinou práve tie časti, ktoré boli súčasťou starého riešenia.
Pre účely testovania bolo rozdelené do koncových bodov, tie zase na funkcie.
Nad každým parametrom bol prevedený rozklad podľa odpovedajúcej charakteristiky a tieto triedy parametrov boli spolu kombinované, aby bolo splnené kritérium pokrytia všetkých kombinácii (ACoC).

\section{Pokračovnaie práce}
Spolu s vývojármi systému Fitcrack je potreba presnejšie zadefinovať špecifikáciu rôznych modulov a s pokračovaním vývoja systému rozšírovať testovacie sady.
Najjednoduchším spôsobom ako zlepšiť testovanie generátora je rozšíriť pseudokód a následne pridať testovacie prípady pre pridané bázové bloky, tým pádom bude jednoducho dohľadateľné, či sa konkrétna funkcionalita generátora testuje.

V budúcnosti by bolo dobré rozšíriť testovaciu sadu aplikačného rozhrania o testovacie prípady aj pre rozšírené časti API.
%TODO: ta dačo lepšie tu dopíš, ne
